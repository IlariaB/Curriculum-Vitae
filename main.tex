%%%%%%%%%%%%%%%%%
% This is an sample CV template created using altacv.cls
% (v1.1.5, 1 December 2018) written by LianTze Lim (liantze@gmail.com). Now compiles with pdfLaTeX, XeLaTeX and LuaLaTeX.
%
%% Edited by Ilaria Battiston
%%
%% It may be distributed and/or modified under the
%% conditions of the LaTeX Project Public License, either version 1.3
%% of this license or (at your option) any later version.
%% The latest version of this license is in
%%    http://www.latex-project.org/lppl.txt
%% and version 1.3 or later is part of all distributions of LaTeX
%% version 2003/12/01 or later.
%%%%%%%%%%%%%%%%

%% If you need to pass whatever options to xcolor
\PassOptionsToPackage{dvipsnames}{xcolor}

%% If you are using \orcid or academicons
%% icons, make sure you have the academicons
%% option here, and compile with XeLaTeX
%% or LuaLaTeX.
% \documentclass[10pt,a4paper,academicons]{altacv}

%% Use the "normalphoto" option if you want a normal photo instead of cropped to a circle
% \documentclass[10pt,a4paper,normalphoto]{altacv}

\documentclass[9pt,a4paper,ragged2e]{altacv}

%% AltaCV uses the fontawesome and academicon fonts
%% and packages.
%% See texdoc.net/pkg/fontawecome and http://texdoc.net/pkg/academicons for full list of symbols. You MUST compile with XeLaTeX or LuaLaTeX if you want to use academicons.

% Change the page layout if you need to
\geometry{left=1cm,right=9cm,marginparwidth=6.8cm,marginparsep=1.2cm,top=1cm,bottom=1cm}

% Change the font if you want to, depending on whether
% you're using pdflatex or xelatex/lualatex
\ifxetexorluatex
  % If using xelatex or lualatex:
  \setmainfont{Carlito}
\else
  % If using pdflatex:
  \usepackage[utf8]{inputenc}
  \usepackage[T1]{fontenc}
  \usepackage[default]{lato}
\fi

% Change the colours if you want to
\definecolor{Mulberry}{HTML}{72243D}
\definecolor{SlateGrey}{HTML}{2E2E2E}
\definecolor{LightGrey}{HTML}{666666}
\colorlet{heading}{Sepia}
\colorlet{accent}{Mulberry}
\colorlet{emphasis}{SlateGrey}
\colorlet{body}{LightGrey}

% Change the bullets for itemize and rating marker
% for \cvskill if you want to
\renewcommand{\itemmarker}{{\small\textbullet}}
\renewcommand{\ratingmarker}{\faCircle}

\begin{document}
\name{Ilaria Battiston}
\tagline{16.01.1998}
\personalinfo{%
  % Not all of these are required!
  % You can add your own with \printinfo{symbol}{detail}
  \email{ilaria.battiston@gmail.com}
  \phone{+39 366 417 3719}
  \location{Munich, Germany}
  \linkedin{linkedin.com/in/ilaria.battiston.5}
  \github{github.com/ila}
  %% You MUST add the academicons option to \documentclass, then compile with LuaLaTeX or XeLaTeX, if you want to use \orcid or other academicons commands.
  % \orcid{orcid.org/0000-0000-0000-0000}
}

%% Make the header extend all the way to the right, if you want.
\begin{fullwidth}
\makecvheader
\end{fullwidth}

%% Depending on your tastes, you may want to make fonts of itemize environments slightly smaller
% \AtBeginEnvironment{itemize}{\small}

%% Provide the file name containing the sidebar contents as an optional parameter to \cvsection.
%% You can always just use \marginpar{...} if you do
%% not need to align the top of the contents to any
%% \cvsection title in the "main" bar.

\cvsection[page1sidebar]{Education}

\cvevent{M.Sc.\ in Data Engineering and Analytics}{Technical University of Munich}{October 2019 - present}{}

\divider

\cvevent{B.Sc.\ in Computer Science --- 105/110}{Università degli Studi di Milano Bicocca}{September 2016 -- July 2019}{}

\medskip

\cvsection{Work Experience}

\cvevent{Web Search Evaluator}{Appen}{September 2018 -- November 2019}{Remote}

\medskip

\cvsection{INTERNSHIPS}

\cvevent{Postgres Database Administrator}{GITC Pro}{January 2020 -- April 2020}{Remote}

\divider

\cvevent{Community Manager}{WikiToLearn}{February 2019 -- September 2019}{Remote}

\divider

\cvevent{Healthcare Data Analyst}{Consorzio Milano Ricerche}{October 2018 -- July 2019}{Milan, Italy}

\medskip

\cvsection{EXTRACURRICULARS AND SCHOLARSHIPS}

\cvevent{Student Developer - Google Summer of Code 2020}{Postgres}{May 2020 -- September 2020}{Remote}

\divider

\cvevent{Deutschlandstipendium}{Scholarship}{September 2019 -- present}{Munich, Germany}

\divider

\cvevent{Conference Organiser}{Postgres Europe}{June 2019 -- present}{Remote}

\divider

\cvevent{Student Developer - Google Summer of Code 2019}{Postgres}{May 2019 -- September 2019}{Remote}

\divider

\cvevent{Administrative Committee Member}{UnixMiB (Open Source students association)}{April 2017 -- present}{Milan, Italy}

\divider

\medskip

\clearpage

\end{document}
